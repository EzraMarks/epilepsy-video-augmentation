%%%%%%%%%%%%%%%%%%%%%%%%%%%%%%%%%%%%%%%%%%%%%%%%%%%%%%%%%%%%%%%%%%%%%%%%%%%%%%%%%%%%%%%%%%%%%%%%
%
% CSCI 1430 Project Progress Report Template
%
% This is a LaTeX document. LaTeX is a markup language for producing documents.
% Your task is to answer the questions by filling out this document, then to 
% compile this into a PDF document. 
% You will then upload this PDF to `Gradescope' - the grading system that we will use. 
% Instructions for upload will follow soon.
%
% 
% TO COMPILE:
% > pdflatex thisfile.tex
%
% If you do not have LaTeX and need a LaTeX distribution:
% - Departmental machines have one installed.
% - Personal laptops (all common OS): http://www.latex-project.org/get/
%
% If you need help with LaTeX, come to office hours. Or, there is plenty of help online:
% https://en.wikibooks.org/wiki/LaTeX
%
% Good luck!
% James and the 1430 staff
%
%%%%%%%%%%%%%%%%%%%%%%%%%%%%%%%%%%%%%%%%%%%%%%%%%%%%%%%%%%%%%%%%%%%%%%%%%%%%%%%%%%%%%%%%%%%%%%%%
%
% How to include two graphics on the same line:
% 
% \includegraphics[width=0.49\linewidth]{yourgraphic1.png}
% \includegraphics[width=0.49\linewidth]{yourgraphic2.png}
%
% How to include equations:
%
% \begin{equation}
% y = mx+c
% \end{equation}
% 
%%%%%%%%%%%%%%%%%%%%%%%%%%%%%%%%%%%%%%%%%%%%%%%%%%%%%%%%%%%%%%%%%%%%%%%%%%%%%%%%%%%%%%%%%%%%%%%%

\documentclass[11pt]{article}

\usepackage[english]{babel}
\usepackage[utf8]{inputenc}
\usepackage[colorlinks = true,
            linkcolor = blue,
            urlcolor  = blue]{hyperref}
\usepackage[a4paper,margin=1.5in]{geometry}
\usepackage{stackengine,graphicx}
\usepackage{fancyhdr}
\setlength{\headheight}{15pt}
\usepackage{microtype}
\usepackage{booktabs}

% From https://ctan.org/pkg/matlab-prettifier
\usepackage[numbered,framed]{matlab-prettifier}

\frenchspacing
\setlength{\parindent}{0cm} % Default is 15pt.
\setlength{\parskip}{0.3cm plus1mm minus1mm}

\pagestyle{fancy}
\fancyhf{}
\lhead{Final Project Progress Report}
\rhead{CSCI 1430}
\rfoot{\thepage}

\date{}

\title{\vspace{-1cm}Final Project Progress Report}


\begin{document}
\maketitle
\vspace{-1cm}
\thispagestyle{fancy}

\textbf{Team name:  TEAM ``HERE PLEASE''}

\textbf{Team members: Alexandra Christine Ryan, Ezra Marks, Isabel Lai}\\
\emph{Note:} Once one person uploads the report to Gradescope, please add all other team members to the submission within the Gradescope interface (top right on your submission).

\textbf{TA name: Andrew Park}

\section*{Project}
% \begin{itemize}
%   \item What is your project idea?
%   Our project idea is to 
%   \item What data have you collected?
  
%   \item What software have you built or used?
%   \item What has each team member contributed thus far?
%     Ezra has done research of software to use in evaluation of our final result. This will 
%   \item What intermediate results have you generated?

%   \item What problems have you faced or still have to consider?
%   \item Is there anything that we can do to help? E.G., resources, equipment.
%   \item 2 pages max; feel free to include any media, references, etc.
% \end{itemize}

\begin{itemize}
\item Do you actually have the data?\\
We have the data about the data we need. After reading a paper titled \textit{Automatic Detection of Epileptogenic Video Content} by Nelson Rodrigues, we have developed a list of video sequences that have regulation breaking content:
    \begin{itemize}
    \item Sequence 1: “potNoodles”, advertisement of “Pot Noodle”, a sequence of synthetic images that change quickly [7].
    \item Sequence 2: “YAT”, a sequence of the 25th episode of the anime YAT Anshin! Uchu RyokÃű [8]. 
    \item Sequence 3: “Pokemon”, a sequence of the 38th episode of the 1st season of Pokemon [9], with flashing sequences and saturated red transitions.
    \item Sequence 4: “London Olympic Games” promotional film [10], a small sequence with flash occurrences.
    \item Sequence 5: “White Stripes - Seven Nation Army”, a music video clip [19].
    \item Sequence 6: “eChannel”, a sequence gathered from E!Channel [20].
    \end{itemize}
Sequences 1 through 4 have caused actual seizures in photosensitive epileptic people in the real world, but all of these have broken regulations! So this is a great way to test edge cases, as sequence 5 and 6 are less egregious than 1 through 4.

For cases where there is no potentially triggering content, we can grab random videos.
\item Do you actually have the compute?\\
A portion of our project will be trying to optimize our MatLab code to run at a reasonable frame rate in real-time on an average laptop -- no extra compute needed.

\item Is there code you need but don't have access to?\\
When we met with our TA, he mentioned that he had some resources that he could send us regarding loading videos in MATLAB!

\item Is there an area where you need help?\\
``We are currently at the stage in our MatLab learning journey where we don't need TA help, as everything they could help us with would be too advanced at this point" - Alex \\
We will need to do a lot of introductory MatLab learning on our own, after which we might benefit from some TA MatLab assistance, especially as it related to optimization.

\end{itemize}


\end{document}
