%%%%%%%%%%%%%%%%%%%%%%%%%%%%%%%%%%%%%%%%%%%%%%%%%%%%%%%%%%%%%%%%%%%%%%%%%%%%%%%%%%%%%%%%%%%%%%%%
%
% CSCI 1430 Project Progress Report Template
%
% This is a LaTeX document. LaTeX is a markup language for producing documents.
% Your task is to answer the questions by filling out this document, then to 
% compile this into a PDF document. 
% You will then upload this PDF to `Gradescope' - the grading system that we will use. 
% Instructions for upload will follow soon.
%
% 
% TO COMPILE:
% > pdflatex thisfile.tex
%
% If you do not have LaTeX and need a LaTeX distribution:
% - Departmental machines have one installed.
% - Personal laptops (all common OS): http://www.latex-project.org/get/
%
% If you need help with LaTeX, come to office hours. Or, there is plenty of help online:
% https://en.wikibooks.org/wiki/LaTeX
%
% Good luck!
% James and the 1430 staff
%
%%%%%%%%%%%%%%%%%%%%%%%%%%%%%%%%%%%%%%%%%%%%%%%%%%%%%%%%%%%%%%%%%%%%%%%%%%%%%%%%%%%%%%%%%%%%%%%%
%
% How to include two graphics on the same line:
% 
% \includegraphics[width=0.49\linewidth]{yourgraphic1.png}
% \includegraphics[width=0.49\linewidth]{yourgraphic2.png}
%
% How to include equations:
%
% \begin{equation}
% y = mx+c
% \end{equation}
% 
%%%%%%%%%%%%%%%%%%%%%%%%%%%%%%%%%%%%%%%%%%%%%%%%%%%%%%%%%%%%%%%%%%%%%%%%%%%%%%%%%%%%%%%%%%%%%%%%

\documentclass[11pt]{article}

\usepackage[english]{babel}
\usepackage[utf8]{inputenc}
\usepackage[colorlinks = true,
            linkcolor = blue,
            urlcolor  = blue]{hyperref}
\usepackage[a4paper,margin=1.5in]{geometry}
\usepackage{stackengine,graphicx}
\usepackage{fancyhdr}
\setlength{\headheight}{15pt}
\usepackage{microtype}
\usepackage{booktabs}
\usepackage{hyperref}

% From https://ctan.org/pkg/matlab-prettifier
\usepackage[numbered,framed]{matlab-prettifier}

\frenchspacing
\setlength{\parindent}{0cm} % Default is 15pt.
\setlength{\parskip}{0.3cm plus1mm minus1mm}

\pagestyle{fancy}
\fancyhf{}
\lhead{Final Project Proposal}
\rhead{CSCI 1430}
\rfoot{\thepage}

\date{}

\title{\vspace{-1cm}Final Project Proposal}


\begin{document}
\maketitle
\vspace{-1cm}
\thispagestyle{fancy}

\textbf{Team name: TEAM ``HERE PLEASE''}

\textbf{Team members: Alexandra Christine Ryan, Ezra Marks, Isabel Lai}\\

\section*{Project}
\begin{itemize}

\item What are the skills of the team members? Conduct a skill assessment!
\begin{itemize}
\item All three of us are taking Computer Vision!!! We have knowledge of image pattern detection and normalization.
\item Ezra can supply the Cognitive Science / photosensitive epilepsy knowledge and research
\item We have programmed things before. As this project seems like it will involve programming, this makes us optimistic about the skill match.

\end{itemize}
\item What is your project idea? \\
We will attempt to create a program to help people with photosensitive epilepsy to avoid seizure-triggering content in online video. For us, there seems to be three main aspects of this project:
\begin{enumerate}
    \item Detect potentially triggering content (in live video?)
    \item Obscure the detected flashing images so that they are no longer seizure-inducing (in our dreams, it would be in a smooth way)
    \item Integrate this with online platforms --- make it useful by creating, for example, a Google Chrome extension.
\end{enumerate}

The first two are the bulk of the computer vision portion, so for the purposes of this final project, we will focus on the video detection and obscuring of triggering content. If we can get the third and make it useful, though, that would be amazing :)

\item What data will you use? \\
Videos of flashing images, e.g. infamous ``Pokemon Shock" episode. Generally, online video of flashing images will be useful for testing.
\item What software/hardware will you use? \\
We'll use existing software to verify that our model works, such as the University of Maryland's \href{https://trace.umd.edu/peat}{Photosensitive Epilepsy Analysis Tool} (PEAT), and we might also need to use screen-capturing software (such as Open Broadcaster Software, OBS) if getting data directly from YouTube (to do our analysis in real-time) doesn't end up working out. We won't need any hardware. 

\item Who will do what?
\begin{itemize}
    \item Everyone: work on finding ways to ingest streaming video into Python.
    \item Everyone: find photosensitive epilepsy-inducing videos to test our program on (videos of flashing images)
    \item Ezra is very passionate about the subject and has done extensive research for many years. He'll definitely be able to lead the direction the project goes and provide valuable insight into the cognitive/more biological aspects of photosensitive epilepsy triggering video.
    \item Isabel, Alex, and Ezra will split up different parts of the project --- for example, video detection algorithms have three different components: Flashy Video Detection, Red Transitions Detection, and Patterned Images Detection. This is perfect, as each group member can take one area of the detection.
\end{itemize}


\item How will you know whether you have made progress? What will you measure?\\
Software that detects seizure-triggering content does exist -- the Photosensitive Epilepsy Analysis Tool (PEAT) is one such program. By testing our own output on this existing well-researched tool, we will be able to measure how successfully we are removing seizure-inducing flashes from the video. Note that although this software detects seizure-inducing content in video, it does not do so in real-time and it does not alter the video to remove the flashes, as we hope to.
\item What problems do you foresee or have? \\
1) Integration with online platforms might be difficult (and isn't particularly vision-related) --- one way to get around this would be to implement what we want (real-time video analysis) but off the platforms (essentially, cut the Chrome extension part, but have all of the core functionality)
\item Is there anything that we can do to help? E.G., resources, equipment.
Nope!
\end{itemize}


\end{document}