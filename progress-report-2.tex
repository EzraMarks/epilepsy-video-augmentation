%%%%%%%%%%%%%%%%%%%%%%%%%%%%%%%%%%%%%%%%%%%%%%%%%%%%%%%%%%%%%%%%%%%%%%%%%%%%%%%%%%%%%%%%%%%%%%%%
%
% CSCI 1430 Project Progress Report Template
%
% This is a LaTeX document. LaTeX is a markup language for producing documents.
% Your task is to answer the questions by filling out this document, then to 
% compile this into a PDF document. 
% You will then upload this PDF to `Gradescope' - the grading system that we will use. 
% Instructions for upload will follow soon.
%
% 
% TO COMPILE:
% > pdflatex thisfile.tex
%
% If you do not have LaTeX and need a LaTeX distribution:
% - Departmental machines have one installed.
% - Personal laptops (all common OS): http://www.latex-project.org/get/
%
% If you need help with LaTeX, come to office hours. Or, there is plenty of help online:
% https://en.wikibooks.org/wiki/LaTeX
%
% Good luck!
% James and the 1430 staff
%
%%%%%%%%%%%%%%%%%%%%%%%%%%%%%%%%%%%%%%%%%%%%%%%%%%%%%%%%%%%%%%%%%%%%%%%%%%%%%%%%%%%%%%%%%%%%%%%%
%
% How to include two graphics on the same line:
% 
% \includegraphics[width=0.49\linewidth]{yourgraphic1.png}
% \includegraphics[width=0.49\linewidth]{yourgraphic2.png}
%
% How to include equations:
%
% \begin{equation}
% y = mx+c
% \end{equation}
% 
%%%%%%%%%%%%%%%%%%%%%%%%%%%%%%%%%%%%%%%%%%%%%%%%%%%%%%%%%%%%%%%%%%%%%%%%%%%%%%%%%%%%%%%%%%%%%%%%

\documentclass[11pt]{article}

\usepackage[english]{babel}
\usepackage[utf8]{inputenc}
\usepackage[colorlinks = true,
            linkcolor = blue,
            urlcolor  = blue]{hyperref}
\usepackage[a4paper,margin=1.5in]{geometry}
\usepackage{stackengine,graphicx}
\usepackage{fancyhdr}
\setlength{\headheight}{15pt}
\usepackage{microtype}
\usepackage{booktabs}

% From https://ctan.org/pkg/matlab-prettifier
\usepackage[numbered,framed]{matlab-prettifier}

\frenchspacing
\setlength{\parindent}{0cm} % Default is 15pt.
\setlength{\parskip}{0.3cm plus1mm minus1mm}

\pagestyle{fancy}
\fancyhf{}
\lhead{Final Project Progress Report}
\rhead{CSCI 1430}
\rfoot{\thepage}

\date{}

\title{\vspace{-1cm}Final Project Progress Report}


\begin{document}
\maketitle
\vspace{-1cm}
\thispagestyle{fancy}

\textbf{Team name:  TEAM ``HERE PLEASE''}

\textbf{Team members: Alexandra Christine Ryan, Ezra Marks, Isabel Lai}\\
\emph{Note:} Once one person uploads the report to Gradescope, please add all other team members to the submission within the Gradescope interface (top right on your submission).

\textbf{TA name: Andrew Park}

\section*{Project}
\begin{itemize}
  \item What is your project idea? \\ \newline
  We are creating a program to help people with photosensitive epilepsy avoid seizure-triggering content in videos. There are three main aspects of this project:
  \begin{enumerate}
    \item Detect sections of video containing potentially triggering content (primarily flashing images)
    \item Obscure the detected flashing images so that they are no longer seizure-inducing. We are experimenting with a variety of techniques to diminish the seizure risk while providing a smooth video-watching experience.
    \item Stretch goal: make the program run in real-time, so that a user can begin viewing a video without waiting for the entire video to process.
  \end{enumerate}

  \item What data have you collected? \\ \newline
  Our data collection methods remain the same as in the past; we'll just use already-existing examples of seizure-inducing flashing videos, which are readily available online
  \item What software have you built or used? \\ \newline
  We've started working with OpenCV (having transitioned from python to Matlab and then back to python), and we're working on building a video processor to read in video frames so that we can analyze and manipulate them. At this point, the software we've built collects frames in a buffer and obscures certain regions of the video.
  \item What has each team member contributed thus far? \\ \newline
    Isabel has done research; she found and read papers about the subject of automatic detection of photosensitive epilepsy triggering video. \\
    Alex has read research papers about detection of photosensitive epilepsy triggering video, and worked with Ezra to design the algorithm we will use to detect flashing videos, and brainstorm approaches to obscuring video to make it non-triggering. \\ 
    Ezra has been working on the groundwork for processing video in Python, reading everything in such that we could potentially switch to real-time processing in the future: reading frames into a buffer, passing those frames to a method for processing / altering them, then displaying those frames to the user. \\
  \item What intermediate results have you generated? \\ \newline We've gotten to the point where we can process and alter videos, and managed to do some very simple work with obscuring parts of the video, indicating that our concept will work.  
  \item What problems have you faced or still have to consider?\\
  We originally attempted to use Matlab due to the recommendation of our TA. However, after much experimentation, we found that Matlab is extremely slow --- not suitable for our purposes. We also identified that we want to use the OpenCV python library, which has many useful functions we can use for detection. Thus, we have concluded that we are going to make the switch to Python!
  \item Is there anything that we can do to help? E.G., resources, equipment. \\ \newline
  If we do go for our stretch goal of processing the video live, we could use help with multi-threading in python.
  \item 2 pages max; feel free to include any media, references, etc.
\end{itemize}


\end{document}
